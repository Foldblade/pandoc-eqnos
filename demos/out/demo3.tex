% Cleveref macros
\providecommand{\crefname}[3]{}
\providecommand{\Crefname}[3]{}
\crefname{equation}{EQ.}{eqs.}
\Crefname{equation}{EQUATION}{Equations}

% Cleveref fakery
\providecommand{\plusnamesingular}{}
\providecommand{\starnamesingular}{}
\providecommand{\cref}{\plusnamesingular~\ref}
\providecommand{\Cref}{\starnamesingular~\ref}

The equation for a straight line is
\begin{equation} y = mx + b \label{eq:line}\end{equation} and the
equation for a polynomial is
\begin{equation} y = \sum_{n=0}^{\infty} a_n x^n \label{eq:polynomial}\end{equation}

\renewcommand{\starnamesingular}{EQUATION}\Cref{eq:line} and
\renewcommand{\plusnamesingular}{EQ.}\cref{eq:polynomial} are known to
all first-year math students.

The Fourier series is a little more advanced:

\begin{equation} y = \frac{1}{2}a_0 + \sum_{n=1}^{\infty}a_n\cos(nx)
                      + \sum_{n=1}^{\infty}b_n\cos(nx)
\label{eq:fourier}\end{equation}

Equations \ref{eq:line}--\ref{eq:fourier} are used throughout science
and engineering.

Equations can be left unnumbered if we do not need to refer to them:
\[ y = A e^{-\gamma t}\cos(2\pi f t) \]

It is also possible to number equations generically without planning to
refer to them; e.g.:
\begin{equation} \pi = 3.141592653589793238462643\dots \label{eq:db29bc9a-127f-4435-bf90-23b91693bf17}\end{equation}

